\subsection{Vietnam War}
The Vietnam War is most likely one of the most well-known wars America was involved with.
The origins of the war date back to the end of the second world war in 1945, where Japan withdrew its soldiers from Vietnam, which was split up into the Democratic Republic of Vietnam, under control of Ho Chi Minh, who strove to create a country following Chinese and Soviet Communism~\autocite{vietnam-war-history} and the state of Vietnam under control of the French emperor Bao Dai, who wanted a Vietnam which had close cultural and economic ties to the West~\autocite{vietnam-war-history}.

Both sides of Vietnam had signed a treaty in the year 1945 during the Geneva Convention, which split Vietnam along the 17 degrees North Latitude, also known as 17th Parallel and set Minh in control of the North and Dai in control of the South. This lasted until 1955, where Ngo Dinh Diem, a "strongly anti communist politician"~\autocite{vietnam-war-history}, became President of the Southern part of Vietnam. Due to the cold war and the US hardening policies against any Communist states, the current US President Eisenhower pledged "firm support to Diem and South Vietnam"~\autocite{vietnam-war-history}.

With the formation of the National Liberation Front in 1960, which was meant to be a resistance against Diem and request of a US report team, John F. Kennedy started increasing US Military Aid in South Vietnam based on the Domino Theory~\autocite{vietnam-war-history}, which assumed that once a country falls to communism, many other countries would follow, like a chain of dominoes~\autocite{cold-war-domino-theory}.

The situation escalated in 1965 when Lyndon B. Johnson sent US Combat Troops to Vietnam.