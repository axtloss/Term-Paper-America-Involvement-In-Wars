\subsection{(Big-) Oil}
Big Oil is a term that describes the biggest Oil Companies under public ownership, these  include BP plc, Royal Dutch Shell plc, Exxon Mobile Corporation, Chevron Corporation, ENI SpA and Total SA~\cite{financial-dictionary-big-oil}. Due to these companies being Government controlled, the Government will set regulations or even start Wars, just to benefit their Oil Companies.

The Wars which are started for the acquirement of Oil are usually called "resource wars"~\cite{belfercenter-oil-conflict}, these however are not the only types of war started due to oil. There is the petro-aggression, which is not a war directly started because of oil, but due to the domestic isolation which oil can cause, making them more likely to try out potentially controversial foreign policies~\cite{belfercenter-oil-conflict}. There's also the financing of other groups with profits made from oil, like the Iran giving oil money to the Hezbollah~\cite{ny-times-lebanon-hezbollah-fuel} and wars started over oil market domination, as seen in the US war with Iraq over Kuwait~\cite{britannica-persian-gulf-war}.

However, oil wars are a very controversial topic, there are people who say that oil has no cause in wars, while others say that oil is a very important topic in wars. Often these discussions are made on the example of the wars in Kuwait, to which the US Secretary of Defense Donald Rumsfeld stated that it's nonsense to suggest that the US invasion of Iraq involved oil in any way~\cite{stokes_blood_for_oil}. Even under Political scientists oil or energy in general is rarely mentioned, the few scientists who do focus on the involvement of oil in wars also disagree in the importance of oil, one group argues that resource wars play a major role in wars, while the other group rejects this claim due to the lack of systematic evidence~\cite{fueling-fire-jeff-d}. The main issue with these discussions is that people often fail to check the influence of oil in the country before the war started, as oil may not be the direct cause of the war, but be a chain reaction of other events which eventually led to a war. It is also noteworthy that oil is a very important resource and simply cannot be ignored when analyzing international security. This can also be seen in the aforementioned discussions about the involvement of oil in the Iraq wars, as scientists argue about involvement of oil in the war, rather than the preconditions caused by oil~\cite{fueling-fire-jeff-d}.

\subsubsection{Resource Wars}
A major example of resource wars is the Iraqi invasion of Kuwait in 1990 and the following US Military operation Desert Shield, in which Iraq successfully annexed Kuwait and gained control of 20 percent of oil reserves~\cite{history-iraq-kuwait}. This is more proof that oil is not always the cause of a war, but rather a precondition of war.

Although the goals of Oil Resource Wars are not always only the ownership of more oil, but more Control over the Organization of the Petroleum Exporting Countries, also known as OPEC, which would allow a country to have more control over the worldwide oil price, and as a result of this power also cut countries out of oil trade.

\subsubsection{Oil Market Domination}
While Wars about Oil Market Domination are similar to Resource Wars, they differ in one may point, the wars aren't for personal gain of oil, but rather to make sure that the opposing force has less oil Power~\cite{fueling-fire-jeff-d}. The US fears that countries with a big oil power could artificially raise the Oil price, which OPEC would have to follow with, allowing the country to enrich itself on oil profits.

Oil Market Domination wars are often a cause of Resource wars, a good example is the US operation Desert Shield, which itself was about a fear of Oil Market Domination from Iraq, but was caused by the Iraqi invasion of Kuwait, which, as already mentioned, was a Resource War.